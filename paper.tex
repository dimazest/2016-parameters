\documentclass[11pt]{article}
\usepackage{acl2016}
\usepackage{times}

\usepackage{url}
\usepackage{latexsym}
\usepackage{booktabs}
\usepackage{multirow}
\usepackage{graphicx}
\usepackage{paralist}
\usepackage{mathtools}
\usepackage{dingbat}
\usepackage{subcaption}
\usepackage{balance}
\usepackage{gensymb}
\usepackage{marginnote}
\usepackage{adjustbox}

\makeatletter
\newcommand{\@BIBLABEL}{\@emptybiblabel}
\newcommand{\@emptybiblabel}[1]{}
\makeatother
% \usepackage{hyperref}

\sloppy

% \aclfinalcopy % Uncomment this line for the final submission
\def\aclpaperid{34} %  Enter the acl Paper ID here


\usepackage{color}
\newcommand{\todo}[1]{}
\renewcommand{\todo}[1]{{\color{red} TODO: {#1}}}

%\renewcommand{\baselinestretch}{0.95}

%\setlength\titlebox{5cm}

% You can expand the titlebox if you need extra space
% to show all the authors. Please do not make the titlebox
% smaller than 5cm (the original size); we will check this
% in the camera-ready version and ask you to change it back.


\title{Robust co-occurrence quantification for lexical distributional semantics experiments}

% \author{First Author \\
%   Affiliation / Address line 1 \\
%   Affiliation / Address line 2 \\
%   Affiliation / Address line 3 \\
%   {\tt email@domain} \\\And
%   Second Author \\
%   Affiliation / Address line 1 \\
%   Affiliation / Address line 2 \\
%   Affiliation / Address line 3 \\
%   {\tt email@domain} \\}

\date{}

\newcommand{\BASEURL}{http://example.org}
% \newcommand{\BASEURL}{https://bitbucket.org/dimazest/phd-buildout/raw/tip/notebooks/downloads/compdistmeaning}
\newcommand{\dataurl}[1]{\href{\BASEURL/#1}{\nolinkurl{#1}}}

\newcommand{\p}{\textsuperscript{\textasteriskcentered}}
\newcommand{\pw}{\textsuperscript{\dag}}

\newcommand{\pp}{\textsuperscript\dag}
\newcommand{\ppp}{\textsuperscript\ddag}
\newcommand{\np}{\phantom{\textsuperscript\textasteriskcentered}}

\begin{document}
\def\emnlp/{\textit{KS2013}}
\def\PhraseRel/{PhraseRel}

\def\PMI/{$1 \operatorname{PMI}$}
\def\SPMI/{$1 \operatorname{SPMI}$}
\def\CPMI/{$1 \operatorname{CPMI}$}
\def\SCPMI/{$1 \operatorname{SCPMI}$}

\def\NPMI/{$n \operatorname{PMI}$}
\def\NSPMI/{$n \operatorname{SPMI}$}
\def\NCPMI/{$n \operatorname{CPMI}$}
\def\NSCPMI/{$n \operatorname{SCPMI}$}

\def\logNPMI/{$\log n\operatorname{PMI}$}
\def\logNSPMI/{$\log n\operatorname{SPMI}$}
\def\logNCPMI/{$\log n \operatorname{CPMI}$}
\def\logNSCPMI/{$\log n \operatorname{SCPMI}$}

\maketitle
\begin{abstract}
Previous optimisations of parameters affecting the word-context association measure used in distributional vector space models have focused on high-dimensional vectors; but low-dimensional versions are often required in compositional tasks. We present a systematic study of the interaction of these parameters and vector dimensionality, and derive parameter selection heuristics that achieve performance across different datasets competitive with the results previously reported in the literature.

\end{abstract}

\section{Introduction}
\label{sec:introduction}

\newcite{TACL570} suggested to optimize PMI with a  set of parameters adopted from predictive models of \newcite{mikolov2013efficient}, namely shifting and context distribution smoothing. Their experiments and thus the parameter selection recommendations use highly dimensional vector spaces with word vector  dimensionality of almost \textbf{200K}. As such, recent work on lexical distributional semantics with the results close to the state of the art involve vectors with a considerable dimensionality of hundreds of thousands \cite{baroni-dinu-kruszewski:2014:P14-1,kiela-clark:2014:CVSC,lapesa2014large}.

Contrary to the trend of employing highly dimensional vectors, other work on lexical and compositional distributional semantics employ vectors with much less dimensions, for example, \textbf{2K} \cite{Grefenstette:2011:ESC:2145432.2145580,kartsadrqpl2014,milajevs-EtAl:2014:EMNLP2014}, \textbf{3K} \cite{Dinu:2010:MDS:1870658.1870771,milajevs-purver:2014:CVSC} or \textbf{10K} \cite{polajnar-clark:2014:EACL,Baroni2010nouns}.

Such a mismatch in vector dimensionality selection across lexical and compositional tasks rises a number of questions this paper aims to answer.
\begin{itemize}
\item To what extent does model performance depend on vector dimensionality?
\item Do parameters influence performance of highly dimensional models the same way as lowly dimensional models? Can the findings of \newcite{TACL570} be directly applied to lowly dimensional models?
\item Is it possible to come up with the heuristics for model parameter selection? If so, what are they?
\end{itemize}

To answer these questions we perform a systematic study of distributional models with a rich set of parameters on two lexical datasets: SimLex-999 \cite{hill2014simlex} and MEN \cite{Bruni:2014:MDS:2655713.2655714}.

\section{Parameters}
\label{sec:parameters}

\begin{figure*}
  \centering
    \includegraphics[width=\textwidth]{supplement/figures/SimLex999-interaction-cds}
    \caption{\textbf{Context distribution smoothing.}}
    \label{fig:interaction-cds}
\end{figure*}

%%% Local Variables:
%%% mode: latex
%%% TeX-master: "paper"
%%% End:


\subsection{PMI variants (\texttt{discr})}
\label{sec:pmi-variants}

Co-occurrence weightings schemes based on \emph{Point-wise mutual information} (PMI, Equation~\ref{eq:pmi}) have been successfully applied in distributional semantics (see for example \newcite{J90-1003}, \newcite{Turney:2010:FMV:1861751.1861756} and \newcite{NIPS2014_5477}).
%
\begin{equation}
  \label{eq:pmi}
  \operatorname{PMI}(x, y) = \log\frac{P(x,y)}{P(x)P(y)}
\end{equation}

PMI is not directly applicable to the co-occurrence matrix because non-observed co-occurrences lead to infinite PMI values which in turn make it impossible to compute similarity. A straightforward ``fix'' to this problem is to replace all infinities with zeroes. Later in the paper, we use PMI to refer to a weighting with this fix.

An alternative solution of avoiding infinity is to increment the probability ratio by one:
%
\begin{equation}
  \label{eq:cpmi}
  \operatorname{CPMI}(x, y) = \log\Big( 1 + \frac{P(x,y)}{P(x)P(y)} \Big)
\end{equation}

\begin{table}
  \centering
  \footnotesize
  \begin{tabular}{llc}
    \toprule
    Parameter & Values & $R^2$ \\
    \midrule
    \texttt{freq} & $1$, $n$, $\log n$                         & 0.36 \\
    \texttt{neg} & 0.2, 0.5, 0.7, 1, 1.4, 2, 5                 & 0.19 \\
    \multirow{2}{*}{Dimensionality} & 1K, 2K, 3K, 5K,          & \multirow{2}{*}{0.17} \\
                                    & 10K, 20K, 30K, 40K, 50K                          \\
    \texttt{cds} & \textit{nan}, 1, 0.75                       & 0.12 \\
    \texttt{discr} & PMI, CPMI, SPMI, SCPMI                    & 0.03 \\
    Similarity & Cosine, correlation                           & 0.02 \\
    \multirow{2}{*}{Corpus} & Concatenation of                 & \multirow{2}{*}{---}  \\
                            & ukWac and Wackypedia                                     \\
    Window size & 5                                            & ---  \\
    \bottomrule
  \end{tabular}
  \caption{\textbf{Parameters, values and their ablation on the SimLex-999 dataset.}}
\label{tab:parameters}
\end{table}

%%% Local Variables:
%%% mode: latex
%%% TeX-master: "paper"
%%% End:


\subsection{Shifted PMI (\texttt{neg})}
\label{sec:shifted-pmi}

It has been experimentally shown that negative PMI values do not positively contribute to model performance and worth nullifying \cite{Turney:2010:FMV:1861751.1861756}. Positive PMI extended with an additional additional hyperparameter $k$ (\texttt{neg}) following \newcite{TACL570} is the third PMI variant (abbreviated as SPMI) used in this work:
%
\begin{equation}
  \label{eq:ppmi}
  \operatorname{SPMI_k} = \max (0, \operatorname{PMI}(x, y) - \log k)
\end{equation}
%
We can apply the same idea to CPMI:
%
\begin{equation}
  \label{eq:pcpmi}
  \operatorname{SCPMI_k} = \max (0, \operatorname{CPMI}(x, y) - \log 2k)
\end{equation}

\subsection{Frequency weighting (\texttt{freq})}
\label{sec:frequency-weighting}

Another issue with PMI is its bias towards rare events \cite{TACL570} one way of solving this issue is to weight PMI value by the co-occurrence frequency \cite{Evert05}:
%
\begin{equation}
  \label{eq:lmi}
  \operatorname{LMI}(x, y) = n(x, y)\operatorname{PMI}(x, y)
\end{equation}
%
where $n(x, y)$ is the number of times $x$ was seen together with $y$. For clarity, we refer to $n$ weighted PMIs as \NPMI/, \NSPMI/ \NCPMI/ and \NSCPMI/. When frequency component is set to 1, it does not have an effect giving rise to explicit quantification labels: \PMI/, \SPMI/, \CPMI/ and \SCPMI/.

In addition to $1$ and $n$ frequencies, we experiment with $\log n$ frequency.

\begin{figure*}
  \centering
    \centerline{\includegraphics[width=1.11\textwidth]{supplement/figures/SimLex999-interaction-neg}}
    \caption{\textbf{Shifted PMI.}}
    \label{fig:interaction-neg}
\end{figure*}

%%% Local Variables:
%%% mode: latex
%%% TeX-master: "paper"
%%% End:


\subsection{Context distribution smoothing (\texttt{cds})}
\label{sec:cont-distr-smooth}

Context distribution smoothing (\texttt{cds}) is another parameter shown to influence performance of distributional models \cite{TACL570}. It smoothes the context distribution $P(x)$:
%
\begin{equation}
  \label{eq:cds}
  P_{\alpha}(x) = \frac{n(x)^{\alpha}}{\sum_{c}n(c)^{\alpha}}
\end{equation}
%
We experiment with values of 1 (absence of smoothing) and 0.75 and sometimes refer to such context probability estimation as \emph{local context probability}.

We also estimate probability of a context using the global context appearance in the corpus:
%
\begin{equation}
  \label{eq:cds-nan}
  P(x) = \frac{n(x)}{|C|}
\end{equation}
%
where $|C|$ is the size of the used corpus. We refer to this case as \emph{global context probability}.

\subsection{Vector dimensionality}
\label{sec:vect-dimens}

As context words we select 1K, 2K, 3K, 5K, 10K, 20K, 30K, 40K and 50K most frequent lemmatised nouns, verbs, adjectives and adverbs. All context words are part of speech tagged, but we do not make difference between refined word types, for example, we do not distinguish between transitive and ditransitive verbs.

\subsection{Similarity metric}
\label{sec:similarity-metric}

We test a widely used cosine based similarity, as well as correlation \cite{kiela-clark:2014:CVSC}.

\section{Experiments}
\label{sec:lexical-experiments}

Table~\ref{tab:parameters} lists parameters used in this work and their values. We will derive the parameter selection heuristics by greedily selecting parameters (\texttt{cds}, \texttt{neg} and similarity measure) that lead to the highest average performance for each frequency, discriminativeness and context combination. Figures~\ref{fig:interaction-cds} and \ref{fig:interaction-neg} show such interaction of \texttt{cds} and \texttt{neg} with other parameters, we don not report similarity behavior due to the limited space.

\paragraph{PMI and CPMI}

PMI achieves the highest performance with global context probabilities. CPMI should be used with local context probabilities and \CPMI/ should apply context distribution smoothing with $\alpha = 0.75$.

\paragraph{SPMI}

10K dimensional \SPMI/ is the least sensitive to parameter selection. For models with dimensionality greater than 20K, context distribution smoothing should be used with $\alpha = 0.75$. For models with dimensionality lower than 20K, it is beneficial to use global context probabilities. Shifting also depends on the dimensionality, models with up to 20K dimensions should set $k = 0.7$ and models with a higher number of dimensions should set $k = 5$. Note, however, that there might be a finer grained $k$ selection criteria, which we do not report in order to avoid overfitting.

\begin{figure*}
  \centering
  \begin{subfigure}[t]{0.6\textwidth}
    \includegraphics[width=\textwidth]{supplement/figures/SimLex999-best}
    \caption{\textbf{SimLex-999.} The black line refers to the score of 0.393. \\
      This work: 0.385,
      SGNS: 0.438,
      GloVe: 0.398.
    }
    \label{fig:best-simlex}
  \end{subfigure}
  ~
  \begin{subfigure}[t]{0.37\textwidth}
    \includegraphics[width=\textwidth]{supplement/figures/SimLex999-global-best}
    \caption{\textbf{SimLex-999.}
    }
    \label{fig:global-best-simlex}
  \end{subfigure}

  \begin{subfigure}[t]{0.6\textwidth}
    \includegraphics[width=\textwidth]{supplement/figures/men-best}
    \caption{\textbf{MEN.} The black line refers to the score of 0.745. \\
      This work: 0.765,
      SGNS: 0.774,
      GloVe: 0.729.
    }
    \label{fig:best-men}
  \end{subfigure}
  ~
  \begin{subfigure}[t]{0.37\textwidth}
    \includegraphics[width=\textwidth]{supplement/figures/men-global-best}
    \caption{\textbf{MEN.}
    }
    \label{fig:global-best-men}
  \end{subfigure}

  \caption{\textbf{Best configurations.} Black line shows the best count models reported by \protect\newcite{TACL570}. We also give our best score, SGNS and GloVe numbers from that study for comparison.}
  \label{fig:best}
\end{figure*}

%%% Local Variables:
%%% mode: latex
%%% TeX-master: "paper"
%%% End:


\logNSPMI/ should be used with global context probabilities for models with vector dimensions up to 20K. For more dimensional spaces, smoothing should be applied with $\alpha = 0.75$, which is similar to \SPMI/. Shifting should be applied with $k = 0.5$ for models with less than 20K dimensions and $k = 1.4$ for models with more than 20K dimensions. In contrast to \SPMI/, which might require change of $k$ as the dimensionality increases, $k = 1.4$ is a much robust choice for \logNSPMI/.

\NSPMI/ gives good results with local context probabilities ($\alpha = 1$). Models with less than 20K dimensions should be used with $k = 1.4$, otherwise $k = 5$ should be preferred.

\paragraph{SCPMI}

With \SCPMI/ and vectors with dimensionality less than 20K global context probability should be used and shifting should be set to $k = 0.7$. Otherwise, local context probability should be used with $\alpha = 0.75$ and $k = 2$.

\NSCPMI/ with dimensions less than 20K should using with global context probability and $k = 1.4$. Otherwise, local context probability without smoothing and $k = 5$ is suggested.

\logNSCPMI/ with models with less than 20K dimensions gives the highest average results with global context probabilities and $k = 0.7$. Otherwise, local context probabilities without smoothing should be preferred with $k = 1.4$.

\subsection{Heurisitc evaluation}
\label{sec:heurisitc-evaluation}

We evaluate the models selected using the heuristics described above against the best possible selection for a given dimensionality choice on the SimLex-999 dataset (refer to Figure~\ref{fig:simlex-ppmi-best-simlex}).

Heuristics for \logNPMI/ and \logNCPMI/ pick the best possible models.

The proposed heuristics work well for \logNSPMI/, where performance variance is low, the performance of models selected using the heuristics is not lower than 0.01 point of the best configuration. For \SPMI/ and \NSPMI/ this difference is much higher.

Heuristics for \logNCPMI/ and \CPMI/ follow the best selection, but with a wider gap than the SPMI models. In general $n$ weighted models do not perform as good as others.

Finally, to see whether the heuristics are applicable to other datasets, we contrast the selected models with the best scores on the MEN dataset (see Figure~\ref{fig:simlex-ppmi-best-men}). Again, PMI and CPMI follow the best possible setup and there is a little gap between the chosen models and the best possible selection for SPMI and SCPMI.

\section{Conclusion}
\label{sec:conclusion}

This paper presents a systematic study of co-occurrence quantification focusing of the selection of parameters presented in \newcite{TACL570}. We replicate their recommendation for highly dimensional vector spaces, and show that with appropriate parameter selection it is possible to achieve comparable performance with lowly dimensional spaces. As an outcome of such an analysis we propose a set of model selection heuristics that maximize performance.

In general, model performance depends on vector dimensionality (the best setup with 50K dimensions is better than the best setup with 1K dimensions by 0.03 on SimLex-999). It is beneficial to experiment with dense and unsmoothed lowly dimensional spaces ($k < 1$, global context probability); and sparse and smooth highly dimensional spaces ($k > 1$, $\alpha = 0.75$). However, these heuristics do not guarantee the best possible result for unweighted and $n$-weighted models because of the high variance of the corresponding scores. Based on this we suggest to use \logNSPMI/ or \logNSCPMI/ with the dimensionality of at least 20K to ensure good performance on lexical tasks.

%\bibliographystyle{acl}
% remove publisher, month etc from conf proceedings:
% \bibliographystyle{acl-short}
\bibliographystyle{acl2016}
\balance
\bibliography{references,dmilajevs_publications}

\end{document}
